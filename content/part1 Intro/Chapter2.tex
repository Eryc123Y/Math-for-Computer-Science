\chapterimage{orange2.jpg}
\chapterspaceabove{6.75cm} 
\chapterspacebelow{7.25cm} 
\chapter{Set, Sequence, Function, and Summation}
\section{Set}
    We begin our formal development with basic notions of set theory. Our most primitive notion is that of a set. This notion is so fundamental that we do not attempt to give a precise definition. We think of a set as a collection of distinct objects with a precise description that provides a way of deciding (in principle) whether a given object is in it.

\begin{definition}[Set]
The objects in a set are its elements or members. When \( x \) is an element of \( A \), we write \( x \in A \) and say ``\( x \) belongs to \( A \)". When \( x \) is not in \( A \), we write \( x \notin A \). If every element of \( A \) belongs to \( B \), then \( A \) is a subset of \( B \), and \( B \) contains \( A \); we write \( A \subseteq B \) or \( B \supseteq A \).
\end{definition}

\begin{remark}
By convention, we use the special characters \( \mathbb{N} \), \( \mathbb{Z} \), \( \mathbb{Q} \), \( \mathbb{R} \) to name the sets of natural numbers, integers, rational numbers, and real numbers, respectively. Each set in this list is contained in the next, so we write \( \mathbb{N} \subseteq \mathbb{Z} \subseteq \mathbb{Q} \subseteq \mathbb{R} \).
\end{remark}

Sets Could be expressed in different ways. For sets with limited and few elements, we simply list all the elements in a pair of bracket, such as $A = \{1, 2, 3, 4, 5\}$. For sets with more elements, we can also define a set by description: $A = \{x: x\in \mathbb{Z^+} \text{ and } x \leq5 \}$. For example:
\begin{example}
    The rational number set could be expressed as:
    $$\mathbb{Q} = \{x: x\in \mathbb{R}, x = \frac{p}{q} \text{ where } p, q \in \mathbb{Z},\text{ but } q \neq 0\}$$
\end{example}

\begin{definition}[Equality of Sets]
Sets \( A \) and \( B \) are equal, written \( A = B \), if they have the same elements. The empty set, written \( \emptyset \), is the unique set with no elements. A proper subset of a set \( A \) is a subset of \( A \) that is not \( A \) itself. The power set of a set \( A \) is the set of all subsets of \( A \).
In other words, the complement of \( A \) includes everything that is not in \( A \).
\end{definition}



\begin{definition}[Basic Set operations]
Intersection, union and exclusion.
 \begin{itemize}
  \item the \textbf{intersection} of \( A \) and \( B \),
  \[ A \cap B = \{x : x \in A \text{ and } x \in B\}; \]
  \item the \textbf{union} of \( A \) and \( B \),
  \[ A \cup B = \{x : x \in A \text{ or } x \in B\}; \]
  \item the \textbf{set difference}, \( A \) but not \( B \),
  \[ A \setminus B = \{x : x \in A \text{ and } x \notin B\}. \]

\end{itemize}

The set \( A \setminus B \) is sometimes called the ``relative complement'' of \( B \) in \( A \).

When \( A \cap B = \emptyset \), sets \( A \) and \( B \) are said to be \textbf{disjoint}.
\end{definition}

\begin{definition}[Complement of Set]
The complement of a set \( A \), often denoted as \( \overline{A} \), $\sim A$ or \( A^c \), is defined with respect to a universal set \( U \), which contains all objects under consideration. The complement \( \overline{A} \) consists of all elements in \( U \) that are not in \( A \). Formally, if we have a universal set \( U \) and a subset \( A \subseteq U \), then the complement of \( A \) is given by:
\[ \overline{A} = \{ x \in U \mid x \notin A \} \]


\end{definition}
\begin{definition}[Intervals]
Intervals. When \( a, b \in \mathbb{R} \) with \( a < b \), the closed interval \( [a, b] \) is the set \( \{x \in \mathbb{R} : a \leq x \leq b\} \). The open interval \( (a, b) \) is the set \( \{x \in \mathbb{R} : a < x < b\} \).
\end{definition}


\section{Properties of Sets with Proofs}

\subsection*{Commutative Laws}
\textbf{Union:}
\[
A \cup B = B \cup A
\]
\textbf{Proof:} 
The union of sets \( A \) and \( B \) includes all elements that are in \( A \), in \( B \), or in both. Since the notion of "being in" does not depend on the order, \( A \cup B \) and \( B \cup A \) represent the same set.

\noindent\textbf{Intersection:}
$$A \cap B = B \cap A$$


\textbf{Proof:} 
The intersection of sets \( A \) and \( B \) includes all elements that are both in \( A \) and in \( B \). The order of \( A \) and \( B \) does not affect the elements that are shared between them, hence the equality.

\subsection*{Associative Laws}
\textbf{Union:}
\[
(A \cup B) \cup C = A \cup (B \cup C)
\]
\textbf{Proof:} 
When we take the union of sets, we combine their elements. Grouping does not affect the outcome of the union, thus the union operation is associative.

\textbf{Intersection:}
\[
(A \cap B) \cap C = A \cap (B \cap C)
\]
\textbf{Proof:} 
The intersection operation finds common elements. The grouping of sets does not affect the commonality of elements, so the intersection operation is associative.

\subsection*{Distributive Laws}
\textbf{Intersection distributes over union:}
\[
A \cap (B \cup C) = (A \cap B) \cup (A \cap C)
\]
\textbf{Proof:} 
An element in \( A \cap (B \cup C) \) is in \( A \) and either \( B \) or \( C \). This is the same as the element being in both \( A \) and \( B \), or in both \( A \) and \( C \).

\textbf{Union distributes over intersection:}
\[
A \cup (B \cap C) = (A \cup B) \cap (A \cup C)
\]
\textbf{Proof:} 
An element in \( A \cup (B \cap C) \) is in \( A \), in both \( B \) and \( C \), or in all. This is equivalent to the element being in \( A \) or \( B \), and in \( A \) or \( C \).

\subsection*{De Morgan's Laws}
\textbf{Complement of the union:}
\[
\overline{A \cup B} = \overline{A} \cap \overline{B}
\]
\textbf{Proof:} 
An element not in \( A \cup B \) is neither in \( A \) nor in \( B \), which means it is in both \( \overline{A} \) and \( \overline{B} \).

\textbf{Complement of the intersection:}
\[
\overline{A \cap B} = \overline{A} \cup \overline{B}
\]
\textbf{Proof:} 
An element not in \( A \cap B \) is not in both \( A \) and \( B \), which means it is either in \( \overline{A} \) or in \( \overline{B} \).

\subsection*{Properties of Complements}
\textbf{Union with complement:}
\[
A \cup \overline{A} = U
\]
\textbf{Proof:} 
The set \( A \) together with all elements not in \( A \) constitutes the entire universe \( U \).

\textbf{Intersection with complement:}
\[
A \cap \overline{A} = \emptyset
\]
\textbf{Proof:} 
No element can be both in set \( A \) and not in set \( A \) at the same time, hence the intersection is the empty set.

\begin{definition}[Subset]
    If $A$ and $B$ are sets, we say $A$ is a subset of $B$ if every element of $A$ is also an element of $B$. This is denoted as \(A \subseteq B\)  
\end{definition}
\begin{remark}
    Note that for every set, it is a subset to itself.
\end{remark}
\begin{definition}[Proper Subset]
    If \( B \subset A \), then \( B \) is a proper subset of \( A \).
\end{definition}
For instance, consider the set \( A = \{1, 2, 3\} \) and the set \( B = \{1, 2\} \). In this case, \( B \subset A \), because every element of \( B \) is in \( A \), but \( A \) contains an additional element \( 3 \) that is not in \( B \).
\begin{definition}[Empty Set]
    The \textbf{empty set} is a unique set that contains no elements. It is denoted as \( \varnothing \). This set is important in set theory because it serves as the identity element for the union operation and has properties that are fundamental to the construction of other sets. For example, every set, including the empty set itself, contains the empty set as a subset:

\[ \varnothing \subseteq A \]

for any set \( A \). Furthermore, the intersection of any set with the empty set is the empty set itself:

\[ A \cap \varnothing = \varnothing \]

This highlights the empty set's role in set operations.
\end{definition}

\begin{definition}[Power Set]
    If \( A \) is any set, the \textbf{power set} of \( A \),
\[ \mathcal{P}(A) = \{S: S \subseteq A \} \]
// the set of all subsets of \( A \)

For example, if \( A = \{a, b, c\} \) then
\[ \mathcal{P}(A) = \{\varnothing, \{a\}, \{b\}, \{c\}, \{a, b\}, \{a, c\}, \{b, c\}, \{a, b, c\} \}. \]
\end{definition}
\begin{definition}[cardinality]
    The number of elements in a set \( S \) is called the \textbf{cardinality} of \( S \) and denoted by \( |S| \). When this is a finite number, then \( |S| \in \mathbb{N} \), and when \( |S| = n \), we'll say that \( S \) is an \( n \)-set. 
\end{definition}
\begin{definition}[Partition of Sets]
    Each element of \( A \cup B \) is in exactly one of the sets \( A \setminus B \), \( B \setminus A \), and \( A \cap B \). More generally,

subsets \( S_1, S_2, S_3, \ldots, S_k \) of \( T \) form a \textbf{partition} of \( T \) means every element of \( T \) belongs to exactly one of the sets \( S_j \).
\end{definition}
The sets \( S_1 = A \setminus B \), \( S_2 = B \setminus A \) and \( S_3 = A \cap B \) form a partition of \( T = A \cup B \).
 In general, \( S_1 \cup S_2 \cup S_3 \ldots \cup S_k \subseteq T \) because each \( S_j \) is a subset of \( T \).
 \( T \subseteq S_1 \cup S_2 \cup S_3 \ldots \cup S_k \) because each element of \( T \) is in some subset \( S_j \).
 Therefore, \( T = S_1 \cup S_2 \cup S_3 \ldots \cup S_k \).

The subsets in a partition are \textbf{mutually disjoint}; that is, any two are disjoint sets.

 If \( p \neq q \), \( S_p \cap S_q = \emptyset \) because no element of \( T \) belongs to more than one \( S_j \).

When \( S_1, S_2, S_3, \ldots, S_k \) forms a partition of \( T \), then
\[ |T| = |S_1| + |S_2| + |S_3| + \ldots + |S_k|. \]


\begin{theorem}[The Principle of Inclusion-Exclusion]
    For any pair of sets,\[ |A \cup B| = |A| + |B| - |A \cap B|, \]

and when \( A \) and \( B \) are disjoint,

\[ |A \cup B| = |A| + |B|. \quad \text{// since } A \cap B = \emptyset \]

Furthermore, we always have

\[ |A \cup B| = |A \setminus B| + |B \setminus A| + |A \cap B|. \]
\end{theorem}

\begin{definition}[The Cartesian Product]
    \textbf{The Cartesian product} of sets \( A \) and \( B \), named for Ren\'e Descartes (1596--1650), is

\[ A \times B = \{(a, b): a \in A \text{ and } b \in B\}, \]

where \( (a,b) \) denotes an ordered pair of objects; there is a first entry and a second entry in each ordered pair. \emph{Parentheses} indicate that order matters.

% Braces indicate it doesn't.
\end{definition}
\begin{remark}
    \[ \{0, 1\} = \{1, 0\}, \text{ but } (0, 1) \neq (1, 0). \quad \text{-- in sets, order doesn't matter; in ordered pairs it does.} \]
    
\[ \{1, 1\} = \{1\}, \text{ but } (1, 1) \neq (1). \quad \text{-- in sets, repetitions don't matter; in ordered pairs they do.} \]

\end{remark}
\begin{example}
    If \( A = \{1,3,5,7\} \) and \( B = \{2,3,5\} \), then

\[ A \times B = \{(1, 2), (1, 3), (1, 5), (3, 2), (3, 3), (3, 5), (5, 2), (5, 3), (5, 5), (7, 2), (7, 3), (7, 5)\} \]
\[ B \times A = \{(2, 1), (2, 3), (2, 5), (2, 7), (3, 1), (3, 3), (3, 5), (3, 7), (5, 1), (5, 3), (5, 5), (5, 7)\}. \]

% (A × B) ∩ (B × A) = {(3,3), (3,5), (5,3), (5,5)}, so A × B ≠ B × A.
\[ (A \times B) \cap (B \times A) = \{(3,3), (3,5), (5,3), (5,5)\}, \text{ so } A \times B \neq B \times A. \]
\end{example}

\subsection{Exercises}
\begin{exercise}
Indicate whether each statement is true or false:
\begin{enumerate}
    \item[(a)] \(\{4, 0, 3, 0\} = \{4, 4, 0, 3\}\)
    \item[(b)] \(\{4\} \subseteq \{0, 3, 4\}\)
    \item[(c)] \(\{0, 3, 4\} \subseteq \{4\}\)
    \item[(d)] \(\{0, 3, 4\} \subseteq \{0, 3, 4\}\)
    \item[(e)] \(\emptyset \subseteq \{0, 3, 4\}\)
\end{enumerate}

\end{exercise}

\begin{exercise}
What is \(\mathcal{P}(\{0, 3, 4, 7\})\)?
\end{exercise}

\begin{exercise}
Let \( A = \{1, 2, 3, 4\} \) and \( B = \{2, 3, 5, 8\} \). Evaluate each of the following expressions:
\begin{enumerate}
    \item[(a)] \( A \cap B \)
    \item[(b)] \( A \cup B \)
    \item[(c)] \( A \setminus B \)
    \item[(d)] \( A \times B \)
\end{enumerate}
\end{exercise}

\begin{exercise}
     Is \(\{1, 3\}, \{2, 3\}, \{4\}\) a partition of \(\{1, 2, 3, 4\}\)? Justify your answer.
\end{exercise}

\begin{exercise}
    Consider the set \(\{a, b, c, d, e\}\). Construct 3 different partitions of this set.
\end{exercise}
Hint: Each partition satisfies the definition: the subsets are non-empty, they cover the entire original set, and they are mutually exclusive (no element is repeated in the subsets of any given partition).

Partition 1:
\begin{itemize}
    \item \(\{a\}, \{b\}, \{c\}, \{d\}, \{e\}\)
\end{itemize}

Partition 2:
\begin{itemize}
    \item \(\{a, b\}, \{c, d, e\}\)
\end{itemize}

Partition 3:
\begin{itemize}
    \item \(\{a, e\}, \{b, c\}, \{d\}\)
\end{itemize}

\begin{exercise}
    Proof that \( A \cap (B - C) = (A \cap B) - (A \cap C) \)
\end{exercise}
\begin{proof}
     To prove the lemma, we will show that the left-hand side (LHS) is a subset of the right-hand side (RHS) and vice versa.

\textit{Using De Morgan's laws:}
\begin{align*}
    \sim (A \cap C) &= \sim A \cup \sim C \\
    \therefore (A \cap B) \sim (A \cap C) &= (A \cap B) \cap (\sim A \cup \sim C)
\end{align*}

\textit{Distributing the intersection over the union:}
\begin{align*}
    (A \cap B) \cap (\sim A \cup \sim C) &= (A \cap B \cap \sim A) \cup (A \cap B \cap \sim C) \\
    &= \varnothing \cup (A \cap B \cap \sim C) \quad \text{(since \( A \cap \sim A = \varnothing \))} \\
    &= A \cap B \cap \sim C
\end{align*}

\textit{Simplifying further:}
\begin{align*}
    A \cap B \cap \sim C &= A \cap (B \cap \sim C) \\
    &= A \cap (B - C) \quad \text{(since \( B \cap \sim C = B - C \))}
\end{align*}

The original statement is proven, as both the LHS and RHS equal \( A \cap (B - C) \).
\end{proof}
\begin{exercise}
   Prove that  \( (A \cup B) - (A \cap B) = (B - A) \cup (A - B) \).
\end{exercise}
\begin{proof}
    We start by applying De Morgan's laws:
\begin{align*}
    (A \cup B) - (A \cap B) &= (A \cup B) \cap \sim (A \cap B) \\
    \text{By De Morgan's laws:} \quad \sim (A \cap B) &= \sim A \cup \sim B \\
    \therefore (A \cup B) \cap \sim (A \cap B) &= (A \cup B) \cap (\sim A \cup \sim B)
\end{align*}

Next, we distribute the intersection over the union:
\begin{align*}
    (A \cup B) \cap (\sim A \cup \sim B) &= ((A \cup B) \cap \sim A) \cup ((A \cup B) \cap \sim B) \\
    &= (\varnothing \cup (A \cap \sim B)) \cup (\varnothing \cup (B \cap \sim A)) \\
    &= (A \setminus B) \cup (B \setminus A)
\end{align*}

Hence, the original statement is proven.
\end{proof}

\begin{exercise}
    Define the set $S = \{x | x = 12m + 8n, m,n \in \mathbb{Z}\}$, and let $P = \{x | x = 20p + 16q, p,q \in \mathbb{Z}\}$.

Prove that $S = P$.
\end{exercise}
\textbf{Proof:}

Let $x \in S$, then $x = 12m + 8n = 4(3m + 2n) = 20(3m + 2n) + 16(-3m - 2n) \in P$, so $S \subseteq P$;

Now let $x \in P$, then $x = 20p + 16q = 4(5p + 4q) = 12(5p + 4q) + 8(-5p - 4q) \in S$,

so $P \subseteq S$; thus, by definition, $S = P$.
%------------------------------------------------
\section{Function: a perspective of Set Theory}
This section discusses function from a perspective of set. We clarify this by the relation between sets.
\begin{definition}[Function]

Let \( A \) and \( B \) be nonempty sets. A function \( f \) from \( A \) to \( B \) is an assignment of exactly one element of \( B \) to each element of \( A \). We write \( f(a) = b \) if \( b \) is the unique element of \( B \) assigned by the function \( f \) to the element \( a \) of \( A \). If \( f \) is a function from \( A \) to \( B \), we write \( f : A \rightarrow B \).
    
\end{definition}
\begin{remark}
    Mapping and transformation are equivalent to function in some context. If \( f \) is a function from \( A \) to \( B \), we say that \( A \) is the \emph{domain} of \( f \) and \( B \) is the \emph{codomain} of \( f \). If \( f(a) = b \), we say that \( b \) is the \emph{image} of \( a \) and \( a \) is a \emph{preimage} of \( b \). The \emph{range}, or \emph{image}, of \( f \) is the set of all images of elements of \( A \). Also, if \( f \) is a function from \( A \) to \( B \), we say that \( f \) \emph{maps} \( A \) to \( B \).
\end{remark}

When we say that two functions are equal, they share the same domain, codomain, and the mapping from the domain to the same codomain.

Think about this problem:
\begin{problem}
    Is $f(x) = \frac{1}{x}$ equal to $f(x) = x^{-1} $?
\end{problem}
Even someone with limited algebra knowledge could know that, even though $\frac{1} {x}$ could be expressed in the same why, it has a different domain to $x$, as for $x \neq 0$ for the first function, while $x\in \mathbb{R}$ for the latter. Their domain and codomain are different.

This example helps to distinguish codomain and range:
\begin{example}
    Let \( f: \mathbb{Z} \rightarrow \mathbb{Z} \) assign the square of an integer to this integer. Then, \( f(x) = x^2 \), where the domain of \( f \) is the set of all integers, the codomain of \( f \) is the set of all integers, and the range of \( f \) is the set of all integers that are perfect squares, namely, \( \{0, 1, 4, 9, \ldots \} \).
\end{example}

\begin{theorem}[Function Addition and Multiplication]
Let \( f_1 \) and \( f_2 \) be functions from \( A \) to \( \mathbb{R} \). Then \( f_1 + f_2 \) and \( f_1f_2 \) are also functions from \( A \) to \( \mathbb{R} \) defined for all \( x \in A \) by
\begin{align*}
    (f_1 + f_2)(x) &= f_1(x) + f_2(x), \\
    (f_1f_2)(x) &= f_1(x)f_2(x).
\end{align*}
\end{theorem}
\begin{problem}
    Let \( f_1 \) and \( f_2 \) be functions from \( \mathbb{R} \) to \( \mathbb{R} \) such that \( f_1(x) = x^2 \) and \( f_2(x) = -x^2 \). What are the functions \( f_1 + f_2 \) and \( f_1f_2 \)?
\end{problem}

\subsection{Injective, Surjective, and Bijective Function}
We have known that functions are actually reflection from one set to the other set. We will look into several special mapping.

\begin{definition}[Injective Function]
    A function \( f: A \to B \) is called \emph{injective} (or \emph{one-to-one}) if every element of the codomain \( B \) is mapped by at most one element of the domain \( A \). For example, the function \( f(x) = 2x \) from \( \mathbb{R} \) to \( \mathbb{R} \) is injective because each value of \( f(x) \) is produced by exactly one value of \( x \).
\end{definition}

\begin{definition}[Surjective Function]
    A function is \emph{surjective} (or \emph{onto}) if every element of the codomain \( B \) is mapped by at least one element of the domain \( A \). For instance, the function \( g(x) = \sin(x) \) from \( \mathbb{R} \) to \( [-1, 1] \) is surjective because every value in \( [-1, 1] \) is the sine of some real number \( x \).
\end{definition}

\begin{definition}[Bijective Function]
    A function is \emph{bijective} if it is both injective and surjective, which means there is a perfect "pairing" between the sets: every element of \( A \) is paired with a unique element of \( B \), and every element of \( B \) is paired with a unique element of \( A \). An example of a bijective function is the identity function \( i(x) = x \) from \( \mathbb{R} \) to \( \mathbb{R} \).
\end{definition}

\begin{figure}[H]
    \centering
    \includegraphics[width=0.75\linewidth]{Images/func.jpg}
    \caption{Examples of Special Mappings}
    \label{fig:fuc}
\end{figure}

Examine the figure \autoref{fig:fuc}, where one example of each type of function is shown. Fundamentally, when we define these functions, only two properties are involve:
\begin{itemize}
    \item A: It the mapping retrievable? (for every value in the image, we can find where it is from without any confusion.
    \item B: Is the mapping is full comparing to the preimage?(Whether every value in the preimage has a mapping to the image)
\end{itemize}
If A is satisfied, we call it injective function.

If B is satisfied, we call it surjective function.

If both A and B are satisfied, we say the function is bijective.

Using $A$, $B$, and $C$ to denote the set of injective, surjective, and bijective function respectively, it is therefore that :

    $$A\cap B = C$$

Which means that if a function is injective and surjective in the same time, it is bijective. 

And these will be all we need to know about function for now, since the rest of the properties will be discussed in single-variable calculus. This section aims only introduce the idea of injectivity, surjectivity, and bijectivity, 

\subsection{Exercises}
\begin{exercise}
    Determine the images of the functions \( f: \mathbb{R} \rightarrow \mathbb{R} \) defined as follows:
    \begin{enumerate}
        \item[a)] \( f(x) = \frac{x^2}{1 + x^2} \).
        \item[b)] \( f(x) = \frac{x}{1 + |x|} \).
    \end{enumerate}
\end{exercise}
\begin{proof}[Solution for a)]
    We analyze the function \( f(x) = \frac{x^2}{1 + x^2} \):
    \begin{itemize}
        \item This function is defined for all \( x \in \mathbb{R} \).
        \item For \( x = 0 \), \( f(0) = 0 \).
        \item For \( x \neq 0 \), \( f(x) \) is always positive.
        \item As \( x \) approaches infinity, \( f(x) \) approaches 1.
    \end{itemize}
    Thus, the image of \( f \) is \( (0, 1] \).
    \end{proof}
    
    \begin{proof}[Solution for b)]
    We analyze the function \( f(x) = \frac{x}{1 + |x|} \):
    \begin{itemize}
        \item This function is defined for all \( x \in \mathbb{R} \).
        \item For \( x > 0 \), as \( x \) increases, \( f(x) \) approaches 1.
        \item For \( x < 0 \), as \( x \) decreases, \( f(x) \) approaches -1.
    \end{itemize}
    Thus, the image of \( f \) is \( (-1, 1) \).
    \end{proof}



%------------------------------------------------
\section{Summation}
Before we move on to the most important part of this chapter, sequence, we use this section to introduce a prerequisite for studying its properties. We introduce the Sigma sign.

In earlier chapters, we have seen exercises such as finding the expression of the sum of the first nth positive integer:
$$1 + 2 + \dots + (n-1) + (n)= \frac{n(n+1)}{2}$$ 
From now on, we will use the sigma notation to deal with the summation of numbers. Such as:
\subsection{Sigma Notation}
\begin{notation}[Sigma Notation]
    \[
\sum_{i=1}^{n} i = \frac{n(n + 1)}{2}
    \]
\end{notation}
Where $i$ is quite similar to iterator, or sometimes we also call counter in programming languages, and $n$ refers to the condition of termination. The expression right after the sigma sign is called \textbf{summand}.

Actually, this is not the only way to express summation, it is also equivalent to:
\[\sum_{1\leq i\leq n} i = \frac{n(n+1)}{2}\]

Also, like what we do for set, we can also write sigma notation using description, such as: 
\[
\sum_{1 \leq k \leq 100} k^2
\]

\( k \) is odd

\subsection{Properties and Techniques of Sigma Notation}
This part of the section shows how we can handle summation expressions.
One of the greatest convenience of sigma notation is that every expression is adjustable, we can change the variable as what we prefer as in the following example.
\begin{example} \label{exp:siginvariance}
 \[
\sum_{1 \leq k \leq n} a_k = \sum_{1 \leq k+1 \leq n} a_{k+1}
\]
\end{example}
This technique has a significant effect to some mathematical proofs.
Another points to keep in mind is that: always make the expression simple in terms of upper and lower boundary.
\begin{example}
    Examine this expression: 
    \[\sum_{k=0}^{n} k(k-1)(k-n)\]
    The sum when k equals to 0, 1, and $n$ is 0. In this case we cannot say it is a good expression, as what we want is the sum it self, while 0 does not matter for us. Therefore, we just fine-tune it to:
    \[\sum_{k=2}^{n-1} k(k-1)(k-n)\]
    This makes it concise and clear.
\end{example}

\subsubsection{Manipulation of Sigma Notation}
For a set \( K \), the following summation properties hold. Let \( c \) be a constant, and \( a_k \), \( b_k \) be sequences indexed by \( K \):

\begin{equation}
    \sum_{k \in K} c a_k = c \sum_{k \in K} a_k \quad \text{(Distributive Law)}
    \label{eq:constant_factor}
\end{equation}

\begin{equation}
    \sum_{k \in K} (a_k + b_k) = \sum_{k \in K} a_k + \sum_{k \in K} b_k \quad \text{(Associative Law)}
    \label{eq:summation_of_sums}
\end{equation}

\begin{equation}
    \sum_{k \in K} a_k = \sum_{p(k) \in K} a_{p(k)} \quad \text{(Commutative Law, as in example \autoref{exp:siginvariance})} 
    \label{eq:permutation_invariance}
\end{equation}
The proof is attached below.

\textbf{Proof of Constant Factor Law:}

Let \( c \) be a constant and \( a_k \) be a sequence indexed by a finite set \( K \). We want to show that \( \sum_{k \in K} c a_k = c \sum_{k \in K} a_k \).

By the definition of summation and the distributive property of multiplication over addition, we have:
\begin{equation}
    \sum_{k \in K} c a_k = c a_1 + c a_2 + \ldots + c a_n = c (a_1 + a_2 + \ldots + a_n) = c \sum_{k \in K} a_k.
\end{equation}
This concludes the proof of the constant factor law.

\textbf{Proof of Summation of Sums Law:}

Let \( a_k \) and \( b_k \) be sequences indexed by a finite set \( K \). We want to show that \( \sum_{k \in K} (a_k + b_k) = \sum_{k \in K} a_k + \sum_{k \in K} b_k \).

By the definition of summation and the associative and commutative properties of addition, we have:
\begin{equation}
    \begin{split}
\sum_{k \in K} (a_k + b_k) &= (a_1 + b_1) + (a_2 + b_2) + \ldots + (a_n + b_n) \\
&= (a_1 + a_2 + \ldots + a_n) + (b_1 + b_2 + \ldots + b_n) \\
&= \sum_{k \in K} a_k + \sum_{k \in K} b_k
\end{split}
\label{eq:sum_of_sums}
\end{equation}
This concludes the proof of the summation of sums law.

\textbf{Proof of Permutation Invariance Law:}

Let \( a_k \) be a sequence indexed by a finite set \( K \). Let \( p: K \to K \) be a bijection, which means \( p \) permutes the indices. We want to show that \( \sum_{k \in K} a_k = \sum_{p(k) \in K} a_{p(k)} \).

By the definition of summation and the fact that addition is commutative (the order does not matter), we have:
\begin{equation}
    \sum_{k \in K} a_k = a_1 + a_2 + \ldots + a_n = a_{p(1)} + a_{p(2)} + \ldots + a_{p(n)} = \sum_{p(k) \in K} a_{p(k)}.
\end{equation}
This concludes the proof of the permutation invariance law.
\subsubsection{Multiple Sums}
Sometimes we use sigma notation with multiple variables, just like what we can do to write loops in programming languages.

    
$$\sum_{1 \leq j, k \leq 3} a_j b_k = a_1b_1 + a_1b_2 + a_1b_3 + a_2b_1 + a_2b_2 + a_2b_3 + a_3b_1 + a_3b_2 + a_3b_3 $$


In the context of summation, we often encounter a situation where a sum is taken over a set of pairs. Specifically, let \( P(j, k) \) be a property involving the indices \( j \) and \( k \), and \( a_{j,k} \) be elements corresponding to these indices. The summation over all pairs \( (j, k) \) satisfying property \( P \) is equivalent to summing over all indices separately:

\[
\sum_{P(j,k)} a_{j,k} = \sum_{j,k} a_{j,k} \cdot P(j,k).
\]

This notation serves as a shorthand for expressing the sum over a subset of indices determined by the property \( P \).
There are also cases where we must use two sigma notation in the same time.

 
\begin{example}[Double Summation]
    When considering a double sum over a set of pairs, we often come across the following identity:

\begin{equation}
\sum_{j}\sum_{k} a_{j,k} [P(j,k)]
\end{equation}

where \( [P(j,k)] \) is an Iverson bracket which equals 1 if the property \( P \) holds for the pair \( (j,k) \) and 0 otherwise.
\end{example}


By interchanging the order of summation, we observe that:

\begin{equation}
\sum_{j}\sum_{k} a_{j,k} [P(j,k)] = \sum_{k}\sum_{j} a_{j,k} [P(j,k)].
\end{equation}

This property allows us to switch the order of summation without changing the result, which can be particularly useful in various mathematical analyses.

Double summation could also be used to simplify a given summation.
Considering the expression at the beginning of this section:
$$\sum_{1 \leq j, k \leq 3} a_j b_k = a_1b_1 + a_1b_2 + a_1b_3 + a_2b_1 + a_2b_2 + a_2b_3 + a_3b_1 + a_3b_2 + a_3b_3 $$

\begin{example}[Converting to Double Summation]
    \[
\sum_{1 \leq i,j,k \leq 3} a_j b_k = \sum_{\substack{i,j,k \\ 1 \leq i,j,k \leq 3}} a_j b_k \left[1 \leq j \leq 3\right] \left[1 \leq k \leq 3\right]
\]
\[
= \sum_j \sum_k a_j b_k \left[1 \leq j \leq 3\right] \left[1 \leq k \leq 3\right]
\]
\[
= \sum_j a_j \left[1 \leq j \leq 3\right] \sum_k b_k \left[1 \leq k \leq 3\right]
\]
\[
= \sum_j a_j \left[1 \leq j \leq 3\right] \left( \sum_k b_k \left[1 \leq k \leq 3\right] \right)
\]
\[
= \left( \sum_j a_j \left[1 \leq j \leq 3\right] \right) \left( \sum_k b_k \left[1 \leq k \leq 3\right] \right)
\]
\[
= \left( \sum_{j=1}^3 a_j \right) \left( \sum_{k=1}^3 b_k \right).
    \]

\end{example}
To explicit:
In the situation where we perform the same range of summation over each variable, the first two lines' triple summation is:
\[
(a_1b_1 + a_1b_2 + a_1b_3) + (a_2b_1 + a_2b_2 + a_2b_3) + (a_3b_1 + a_3b_2 + a_3b_3).
\]
Utilizing the distributive property to combine the summation operations into one involving \( a \), since \( a \) and each \( k \) for \( 1 \leq j \leq 3 \) are independent, yields (as in the third line):
\[
a_1(b_1 + b_2 + b_3) + a_2(b_1 + b_2 + b_3) + a_3(b_1 + b_2 + b_3).
\]

Consider a double sum over two independent indices, if the indices are independent, the summation of the product can be split into the product of two summations. For instance, the sum of products of \(a_j\) and \(b_k\) over \(j\) in \(J\) and \(k\) in \(K\) can be expressed as: \( (a_1 + a_2 + a_3)(b_1 + b_2 + b_3) \). This can be generalized to an expression:

\begin{equation}
\sum_{j \in J} \sum_{k \in K} a_j b_k = \left( \sum_{j \in J} a_j \right) \left( \sum_{k \in K} b_k \right),
\end{equation}
which is known as the \textbf{general distributive law}.
\begin{remark}
    If you are an agile reader, you must have noticed that this expression is a kind of representation of Cartesian sets in algebra. The general distributive law allows the sum over a function of elements from the Cartesian product of two sets to be expressed as the product of sums over each set if the function is separable into independent factors.
\end{remark}
\subsection{Exercises}
\begin{exercise}
    Express the triple sum
\[
\sum_{1 \leq i < j < k \leq 4} a_{ijk}
\]
as a three-fold summation (with three \(\sum\)'s), 

\begin{enumerate}[label=\alph*.]
    \item summing first on \( k \), then \( j \), then \( i \);
    \item summing first on \( i \), then \( j \), then \( k \).
\end{enumerate}
Also write your triple sums out in full without the \(\sum\)-notation, using parentheses to show what is being added together first.
\end{exercise}
\textbf{Solution:}

\textbf{(a)} \[
\sum_{i=1}^{4} \sum_{j=i+1}^{4} \sum_{k=j+1}^{4} a_{ijk} = 
\sum_{i=1}^{2} \sum_{j=i+1}^{3} \sum_{k=j+1}^{4} a_{ijk} = 
((a_{123} + a_{124}) + a_{134}) + a_{234}.
\]

\textbf{(b)} \[
\sum_{k=1}^{4} \sum_{j=1}^{k-1} \sum_{i=1}^{j-1} a_{ijk} = 
\sum_{k=3}^{4} \sum_{j=2}^{k-1} \sum_{i=1}^{j-1} a_{ijk} = 
a_{123} + (a_{124} + a_{134} + a_{234}).
\]

\begin{exercise}
    Demonstrate your understanding of \(\Sigma\)-notation by writing out the sums
\[
\sum_{k=0}^{5} a_k \quad \text{and} \quad \sum_{0\leq k^2 \leq 5} a_{k^2}
\]
in full. (Watch out—the second sum is a bit tricky.)

\end{exercise}
\textbf{Solution:}

The first sum is:
\[
a_0 + a_1 + a_2 + a_3 + a_4 + a_5
\]

The second sum,  $k \in \{-2, -1, 0, 1, 2\}$, therefore:
\[
a_4 + a_1 + a_0 + a_1 + a_4
\]

\begin{exercise}
    The general rule for summation by parts is equivalent to
\[
\sum_{0 \leq k < n} (a_{k+1} - a_k)b_k = a_nb_n - a_0b_0 - \sum_{0 \leq k < n} a_{k+1}(b_{k+1} - b_k), \quad \text{for } n \geq 0.
\]

Prove this formula by using the distributive, associative, and commutative laws.
\end{exercise}
Hint: Use Associative Law to LHS, try to make the indices of the two sums as similar as possible.
\begin{proof}
    \begin{align*}
        \text{LHS} &= \sum_{0\leq k < n} a_k b_{k+1} - \sum_{0\leq k < n} a_k b_k \\
        &= \sum_{0\leq k < n} a_k b_{k+1} - \sum_{-1\leq k < n-1} a_{k+1} b_{k+1} \\
        &= \sum_{0\leq k < n} a_k b_{k+1} - \sum_{0\leq k < n-1} a_{k+1} b_{k+1} \\
        &= \sum_{k=0}^{n-1} a_k b_{k+1} - \sum_{k=0}^{n-2} a_{k+1} b_{k+1} \\
        &= a_n b_{n-1} - a_0 b_0 + \sum_{k=0}^{n-2} a_k b_k - \sum_{k=0}^{n-2} a_{k+1} b_{k+1} \\
        &= a_n b_{n-1} - a_0 b_0 + \sum_{0\leq k < n-1} a_k (b_k - b_{k+1}) \\
        &= a_n (b_n - b_{n-1}) + a_n b_{n-1} - a_0 b_0 - \sum_{0\leq k < n-1} a_{k+1} (b_{k+1} - b_k) \\
        &= a_n (b_n - b_{n-1} + b_{n-1}) - a_0 b_0 - \sum_{0\leq k < n} a_{k+1} (b_{k+1} - b_k) \\
        &= a_n b_n - a_0 b_0 - \sum_{0\leq k < n} a_{k+1} (b_{k+1} - b_k) \\
        &= \text{RHS}
        \end{align*}
\end{proof}
\begin{exercise}
    Is the following expression correct or not? Give your reason.
    \[
\left( \sum_{i=1}^{n} a_i \right) \left( \sum_{j=1}^{n} \frac{1}{a_j} \right) = \sum_{1 \leq i \leq n} \sum_{1 \leq j \leq n} \frac{a_i}{a_j} = \sum_{1 \leq i \leq n} \sum_{1 \leq i \leq n} \frac{a_i}{a_i} = \sum_{i=1}^{n} 1 = n
\]
\end{exercise}
\textbf{Solution:}

Consider the expression given by:
\[
\left( \sum_{i=1}^{n} a_i \right) \left( \sum_{j=1}^{n} \frac{1}{a_j} \right)
\]
and its expansion into a double sum:
\[
\sum_{1 \leq i \leq n} \sum_{1 \leq j \leq n} \frac{a_i}{a_j}
\]

It is claimed that this is equal to:
\[
\sum_{1 \leq i \leq n} \sum_{1 \leq i \leq n} \frac{a_i}{a_i} = \sum_{i=1}^{n} 1 = n
\]

However, this claim overlooks the fact that the double sum includes terms where \( i \neq j \), which are not necessarily equal to 1. Only when \( i = j \) does the term \( \frac{a_i}{a_j} \) simplify to 1, contributing to the count of \( n \).

Hence, the proper expansion of the double sum should be written as:
\[
\sum_{i=1}^{n} \sum_{j=1}^{n} \frac{a_i}{a_j} = \sum_{i=1}^{n} 1 + \sum_{\substack{i,j=1 \\ i \neq j}}^{n} \frac{a_i}{a_j}
\]
where the first sum on the right-hand side counts the \( n \) instances where \( i = j \), and the second sum accounts for the \( n(n-1) \) instances where \( i \neq j \).

The claim would only be true if all \( a_i \) are equal, which is a special case, not the general case. In the general case, the expression evaluates to something different than \( n \) due to the presence of terms where \( i \neq j \). 

Therefore, the original statement is incorrect unless the condition that all \( a_i \) are equal is specified.

\begin{exercise}
    Consider the following double summation where $a_i, a_j, b_i, b_j \in \mathbb{R}$.
    $$\sum_{i=1}^{n} \sum_{j=1}^{n} (a_ib_j-a_jb_i)$$
    $$\sum_{i=1}^{n} \sum_{j=1}^{n} (a_ib_j-a_jb_i)^2$$
    Is there anything special about these expressions? Manage to find all the equivalent expressions of
    the sum of squares in sigma notation. Also consider, if the order of summand increases to infinity, 
    whether these properties still exist?
\end{exercise}
\textbf{Solutions:}

For $\sum_{i=1}^{n} \sum_{j=1}^{n} (a_ib_j-a_jb_i)$
\begin{itemize}
    \item It could be seen that the sums are actually symmetrical. When $i=j$, $a_ib_j-a_jb_i=0$.
    \item If you list several of the first nth term, the term with indices $(i, j)$ will be canceled by $(j, i)$ term, since $(a_ib_j-a_jb_i)+(a_jb_i-a_ib_j)=0$ 
\end{itemize} 
We can visualize it in a matrix with $n=5$.
\begin{figure}[H]
    \centering
    \includegraphics[width = 0.8\textwidth]{mtxvis.png}
    \caption{Visualization of $\sum_{i=1}^{n} \sum_{j=1}^{n} (a_ib_j-a_jb_i)$}
\end{figure}

With this image, we expand it until n; we can still cancel all elements symmetrical by the diagonal on by one. As the sum on the diagonal is 0, we conclude that 
$$\sum_{i=1}^{n} \sum_{j=1}^{n} (a_ib_j-a_jb_i)=0$$

Now consider the summation of squares. $(a_ib_j-a_jb_i)^2 = 0$ still holds for  $i=j$, but what about the symmetrical pairs $(i, j)$ and $(j, i)$?
We can figure it out by analysis by expanding the square of sum.
$$\sum_{i=1}^{n} \sum_{j=1}^{n}\left(a_{i}^{2} b_{j}^{2}-2 a_{i} a_{j} b_{i} b_{j}+a_{j}^{2} b_{i}^{2}\right)$$
By associative property of summation, we rearrange it as:
$$\sum_{i=1}^{n} \sum_{j=1}^{n}(a_i^2 b_j^2+a_j^2b_i^2) - 2\sum_{i=1}^{n} \sum_{j=1}^{n}a_ia_jb_ib_j$$
When $i \neq j$, each pair of $(i, j)$ and $(j, i)$. The sum of symmetric pair is
$$(a_i^2 b_j^2+a_j^2b_i^2 - 2a_ia_jb_ib_j) + (a_j^2 b_i^2+a_i^2b_j^2 - 2a_ia_jb_ib_j)$$
Rearrange it as:
$$2(a_i^2 b_j^2+a_j^2b_i^2) - 4(a_ia_jb_ib_j)$$
Still, as the sum of $(i, j)$ terms where $i=j$ is 0. We can ignore the diagonal. Hence, we have $n/2$ pairs
of $(a_i^2 b_j^2+a_j^2b_i^2) - 4(a_ia_jb_ib_j)$. This could be written as:
$$\frac{1}{2} \sum_{i=1}^{n} \sum_{j=1}^{n}2[(a_i^2 b_j^2+a_j^2b_i^2) - 4(a_ia_jb_ib_j)]$$
$$\sum_{i=1}^{n} \sum_{j=1}^{n}[(a_i^2 b_j^2+a_j^2b_i^2) - 2(a_ia_jb_ib_j)]$$
Notice that, $\sum_{i=1}^{n} \sum_{j=1}^{n}a_i^2 b_j^2 = \sum_{i=1}^{n} \sum_{j=1}^{n}a_j^2b_i^2$ due to the
diagonal symmetry of the summation as illustrated in the graph. We are adding the same term twice. 
Hence, we have:
$$2\sum_{i=1}^{n} \sum_{j=1}^{n}[a_i^2 b_j^2 - (a_ia_jb_ib_j)]$$
We can further simplify it by rule out the terms where $i=j$, as the summand is 0 in those cases.
We rewrite the sum as:
$$2\sum_{i=1}^{n-1} \sum_{j=2}^{n} [a_i^2 b_j^2 - (a_ia_jb_ib_j)]$$
or in single summation, we have:
$$2\sum_{i \neq j}[a_i^2 b_j^2 - (a_ia_jb_ib_j)] = 2\sum_{1 \leq i < j \leq n}[a_i^2 b_j^2 - (a_ia_jb_ib_j)] $$

For $\sum_{i=1}^{n} \sum_{j=1}^{n} (a_ib_j-a_jb_i)^n$, the symmetric property of summand still exists.
However we cannot prove it for now, as we need to use polynomial theorem to be introduced in Combinatorics.

%------------------------------------------------

\section{Sequence}
If everything so far is not a problem for you, congratulations, because you have known everything you need to know sequence. Sequence is an important concept that will be used throughout your journey of learning math. A Sequence
is defined as:
\begin{definition}
    A sequence is a function \( f: \mathbb{N} \to \mathbb{R} \), where \(\mathbb{N}\) is the set of natural numbers and \(\mathbb{R}\) is the set of real numbers. The value \( f(n) \) 
    is the \( n \)-th term of the sequence, often denoted as \( a_n \). Therefore, a sequence can be represented as \( \{a_n\}_{n=1}^{\infty} \) for an infinite sequence or \( \{a_n\}_{n=1}^{N} \) for a finite sequence of length \( N \).
\end{definition} 

\subsection{Introduction}
Sequences could be either infinite or finite. A \textit{sequence} is defined to be a function \( S \) whose domain \( D \) is a nonempty interval of integers. \( S \) is an \textit{infinite sequence} if \( D \) has the form \( \{a..\} \). % Usually \( a \) is 1 or 0.

\( S \) is a \textit{finite sequence} if \( D \) has the form \( \{a..b\} \) where \( a \leq b \). When \( |D| = n \), we will say that \( S \) is an \( n\text{-sequence} \). We will take the domain of an \( n\text{-sequence} \) to be the set \( \{1..n\} \). % But \( a \) could be 0, and then \( D \) is \( \{0..(n - 1)\} \).

The (natural) ordering of the domain of a sequence \( S \) gives a natural ordering to the ordered pairs in the set \( S \). If \( S \) is a 5-sequence, then
\[ S = \{(1, S(1)), (2, S(2)), (3, S(3)), (4, S(4)), (5, S(5))\} \]

\begin{example}
    Suppose \( D = \{1..10\} \), and we define the function \( S \) on \( D \) by

\[ S(i) = \text{the smallest prime factor of the integer } (1 + i). \]

Then \( D \) is a finite interval of integers, and so \( S \) is the sequence denoted by

\[ S = (2, 3, 2, 5, 2, 7, 2, 3, 2, 11). \]
\end{example}

\begin{definition}[Sum of Sequence]
    If \( S = (S_1, S_2, S_3, \ldots, S_n) \) is a finite sequence of numbers, the corresponding \textit{series} is the sum of the entries in \( S \)
    is:

\[ S_1 + S_2 + S_3 + \ldots + S_n. \]
\end{definition}
\subsection{Special Sequences}
This section introduces common sequences as well as their properties.
\subsubsection{Algorithmic Sequence}
Algorithmic sequences are a fundamental concept in both computer science and mathematics, forming the backbone of algorithm design and analysis. These sequences are typically defined as an ordered set of steps or instructions, aimed at solving a specific problem or accomplishing a particular computation.
\begin{definition}[Arithmetic Sequence]
    An arithmetic sequence of the form
\[
a, a + d, a + 2d, \ldots, a + nd, \ldots
\]
where the initial term \( a \) and the common difference \( d \) are real numbers.
\end{definition}
Usually, the notation $a_n$ is used to express the nth term of a sequence (starting from 0). For the example in the 
definition, we have $a_0 = a$ and $a_n = a_0 + nd$. 
We also have:
$$a_1 = a_0 + d$$
$$a_2 = a_1 + d$$
$$\dots$$
$$a_n = a_0 + nd$$ 
for $n>=1$, $n\in \mathbb{Z}$:
$$a_n = a_{n-1} + d$$
These formula shows the linking between consecutive terms in an arithmetic sequence. We know that the sum
$s$ of the sequence is:
\begin{align*}
    S & =\ a_{0} \ +\ a_1\ \ +\ \dotsc \ +\ a\_n\\
    S & =\ a_{0} \ +\ a_{0} +d\ +\ \dotsc \ +\ a_{n-1} +d
    \end{align*}
The sum is expressed in infinite terms, and it is called \textbf{open form equation}. Accordingly, there are also \textbf{closed form
equations}.
\begin{definition}[Open Form]
    An \textit{open form} or \textit{non-closed form} expression, on the other hand, does not have a finite standard representation and often requires recursive or iterative methods for evaluation. It may involve summations, integrals, or other operations that are not easily simplified into a finite number of operations.
\end{definition}
\begin{definition}[Closed Form]
    A \textit{closed form} expression is a mathematical expression that can be evaluated in a finite number of standard operations. It typically involves constants, variables, and operations from algebra, calculus, and other areas of mathematics that can be computed in a finite number of steps.
    A closed form expression provides a direct way to compute the term of a sequence without the need for recursion.
\end{definition}

Is the open form good for calculating the sum of a sequence? Suppose now I want to know $S_{100}$ (The sum of the first 100th terms),
with the open form, I still have to calculate 99 terms using the definition of this sequence. So is there a way 
to make it possible that we get the sum in one step? Think about closed form. The closed form allows us to calculate
the sum directly. But is it possible to transform an open expression to closed form? If possible, how?

You may already notice that the open form has a property of infinity, and each step is somewhat related.
Isn't it a perfect problem to be solved by mathematical induction? We will leave this proof as a exercise, and here we provide another direct proof
by the symmetry of arithmetic sequence.
\begin{theorem}[Sum of Arithmetic Sequence]
    For  arithmetic sequence \( a_0, a_1, \ldots, a_{n-1} \), where each term can be expressed as \( a_i = a_0 + id \) and \( d \) is the common difference. The sum of the first nth terms is:
    \[ S = \frac{n}{2}[2a_0 + (n-1)d] \] or \[ S = \frac{n}{2}(a_0 + a_{n-1}) \] where \[ a_n = a_0 + nd \]
\end{theorem}
\begin{remark}
    We are trying to find the sum of the first n terms, and the first term is $a_0$, so the last term is
    $a_{n-1}$.
\end{remark}
\begin{proof}
    Consider an arithmetic sequence \( a_0, a_1, \ldots, a_n \), where each term can be expressed as \( a_i = a_0 + id \) and \( d \) is the common difference.
    \item Write the sum of the sequence in order:
\[
S = a_0 + (a_0 + d) + (a_0 + 2d) + \ldots + (a_0 + (n-1)d)
\]

\item Write the sum of the sequence in reverse order:
\[
S = (a_0 + (n-1)d) + (a_0 + (n-2)d) + \ldots + a_0
\]

\item Add these two equations together, every pair of terms within the brackets forms: \[ 2a_0 + (n-1)d \]

\item Since each term appears in a pair, there are \( n \) such pairs.

\item The resulting equation is \( 2S = n[2a_0 + (n-1)d] \).

\item Solving for \( S \) gives us \( S = \frac{n}{2}[2a_0 + (n-1)d] \) or \( S = \frac{n}{2}(a_0 + a_n) \), where \( a_n = a_0 + (n-1)d \).
\end{proof}


\subsubsection{Geometric Sequence}
Geometric sequence is the other important and common sequence that involved in problem-solving of computer
Science. 
A geometric sequence, also known as a geometric progression, is a sequence of numbers where each term after 
the first is found by multiplying the previous term by a fixed, non-zero number called the common ratio. 
Mathematically, a geometric sequence is defined as follows:
\begin{definition}[Geometric Sequence]
    Given the first term \( a_0 \) (also referred to as \( a_1 \) in some texts) and the common ratio \( r \), the \( n \)-th term of a geometric sequence \( a_n \) can be expressed as:
\[
a_n = a_0 \cdot r^n \quad \text{for } n \geq 0
\]
where \( n \) is a non-negative integer representing the position of the term in the sequence.
\end{definition}    

The common ratio \( r \) can be any real number. If \( |r| < 1 \), the terms of the sequence 
will get progressively smaller and approach zero. If \( |r| > 1 \), the terms will grow progressively 
larger. If \( r = 1 \), the sequence is constant, and if \( r = -1 \), the sequence will alternate 
between two values.

We can deduce the sum of a specific geometric sequence by direct proof.
\begin{theorem}[Sum of Geometric Sequence]
    
\end{theorem}

\begin{proof}
    Consider a geometric sequence with the first term \( a_0 \) and the common ratio \( r \) where \( r \neq 1 \). The sequence is given by:
\[ a_0, a_0r, a_0r^2, \ldots, a_0r^{n-1} \]

The sum of the first \( n \) terms of the sequence, denoted by \( S_n \), is:
\[ S_n = a_0 + a_0r + a_0r^2 + \ldots + a_0r^{n-1} \]

To find a formula for \( S_n \), multiply the entire sequence by \( r \):
\[ rS_n = a_0r + a_0r^2 + a_0r^3 + \ldots + a_0r^n \]

Subtract the original sum \( S_n \) from this new sum \( rS_n \) to get a telescoping series:
\[ rS_n - S_n = a_0r^n - a_0 \]

Solving for \( S_n \) gives us:
\[ S_n = \frac{a_0(1 - r^n)}{1 - r} = \]

This is the sum formula for the first \( n \) terms of a geometric sequence when \( r \neq 1 \). If \( r = 1 \), the sequence is constant, and the sum of the first \( n \) terms is simply \( n \) times the first term \( a_0 \).
\end{proof}
%----------------------------------------------------------------------
\subsubsection{characteristic Sequence}
\begin{definition}[Characteristic Sequence]
    Suppose that \( U \) is some given \( n \)-set whose elements have been \textit{indexed} (listed in a certain order) so that \( U = \{x_1, x_2, \ldots, x_n\} \). If \( A \) is a subset of \( U \), the \textit{characteristic sequence} of \( A \) is the function whose domain is \( \{1..n\} \) defined by

\[
X^A_i = X^A(i) = 
\begin{cases} 
1 & \text{if } x_i \in A \\
0 & \text{if } x_i \notin A 
\end{cases}
\]
\end{definition}

\begin{example}
    If \( U \) is the set of the first 10 odd positive integers, \( A \) is the subset of primes in \( U \), and \( B \) is the set of multiples of 3 in \( U \), then
    
    \[
    \begin{aligned}
    &U = \{1, 3, 5, 7, 9, 11, 13, 15, 17, 19\} &&// x_i = 2i - 1. \\
    &A = \{3, 5, 7, 11, 13, 17, 19\} \\
    &B = \{3, 9, 15\} \\
    &X^A = (0, 1, 1, 1, 0, 1, 1, 0, 1, 1) \\
    &X^B = (0, 1, 0, 0, 1, 0, 0, 1, 0, 0).
    \end{aligned}
    \]
    \end{example}

    Characteristic sequences may be used as an implementation model for subsets of any given indexed set \( U \). The set operations may be done on these sequences:
    \[
    X^{A \cap B}_i = X^A_i \times X^B_i;
    \]
    \[
    X^{A \cup B}_i = X^A_i + X^B_i - X^A_i \times X^B_i;
    \]
    \[
    X^{A \setminus B}_i = X^A_i - X^A_i \times X^B_i.
    \]
    
    If \( A \subseteq B \) then
    \[
    X^A_i \leq X^B_i \quad \text{for each index } i,
    \]
    and
    \[
    |A| = \sum_{i=1}^{n} X^A_i.
    \]

%----------------------------------------------------------------------
\subsection{Exercises}
\begin{exercise}
    Find the sum of arithmetic sequence using mathematical induction. Try \textbf{NOT} use the conclusion in this section.
\end{exercise}
Hint: Consider the sum of the first nth positive integer. Try to make assumption by taking it as an arithmetic sequence.
\begin{proof}
    Let \( S(n) \) denote the sum of the first \( n \) terms of an arithmetic sequence with the first term \( a_0 \) and common difference \( d \).
    \begin{itemize}
        \item \textbf{Base Case (\( n = 1 \))}: The sum of the sequence with only the first term is the first term itself, \( S(1) = a_0 \).
        \item \textbf{Inductive Step}: Assume that the sum of the first \( k \) terms \( S(k) \) is given by a certain formula. We want to show that the sum of the first \( k+1 \) terms \( S(k+1) \) can be expressed using the same formula.
        \end{itemize}
        
        For the base case, we can easily see that:
        \[
        S(1) = a_0
        \]
        As $\sum_{1}^{n}i = \frac{n(n+1)}{2}$, which could be taken as an arithmetic sequence with $a_0=1$ and
        $a_n=n$.
        By this, assume that the sum of the first \( k \) terms is:
        \[
        S(k) = \frac{k}{2} [a_0 + a_n]
        \]
        equivalent to
        \[
        S(k) = \frac{k}{2} [2a_0 + (k-1)d]
        \]
        
        To prove the inductive step for \( S(k+1) \), consider:
        \[
        S(k+1) = S(k) + a_0 + kd
        \]
        
        Substituting the inductive hypothesis into the above equation yields:
        \[
        S(k+1) = \frac{k}{2} [2a_0 + (k-1)d] + a_0 + kd
        \]
        
        After simplifying, we aim to show that:
        \[
        S(k+1) = \frac{k+1}{2} [2a_0 + kd]
        \]
        
        This will complete the proof if we can establish that the simplified version of \( S(k+1) \) matches the form of the inductive hypothesis.
\end{proof}
\begin{exercise}
    Prove the sum of geometric sequence  is \( S_n = \frac{a_0(1 - r^n)}{1 - r} = \) using mathematical induction.
\end{exercise}
\begin{proof}
    We want to prove that the sum of the first \( n \) terms of a geometric sequence \( S_n \) with the first term \( a \) and common ratio \( r \) (where \( r \neq 1 \)) is given by:

\[ S_n = \frac{a(1 - r^n)}{1 - r} \]

\textbf{Base Case (n=1):}

The sum of the first term is simply the term itself:

\[ S_1 = a \]

which agrees with the formula.

\textbf{Inductive Step:}

Assume the formula holds for \( n = k \), that is,

\[ S_k = \frac{a(1 - r^k)}{1 - r} \]

We need to prove that it also holds for \( n = k+1 \):

\[ S_{k+1} = \frac{a(1 - r^{k+1})}{1 - r} \]

Starting with the inductive hypothesis for \( S_k \) and adding the \( (k+1) \)-th term \( ar^k \) to both sides, we have:

\[ S_k + ar^k = \frac{a(1 - r^k)}{1 - r} + ar^k \]

Simplifying, we obtain:

\[ S_{k+1} = S_k + ar^k = \frac{a - ar^{k+1}}{1 - r} \]

which is the same as the formula for \( S_{k+1} \), thus completing the proof.
\end{proof}
\begin{exercise}
    Given a sequence \( \{a_n\} \) and a series \( S_n = an^2 + bn + c (a \neq 0) \).

\begin{enumerate}
    \item Find the general term \( a_n \);
    \item Is the sequence \( \{a_n\} \) an arithmetic sequence?
\end{enumerate}
\end{exercise}
Hint: How can we get the value of a term from the sum of a sequence?
\textbf{Solution:}

\begin{enumerate}
    \item For \( n \geq 2 \), \( a_n = S_n - S_{n-1} = (an^2 + bn + c) - [a(n-1)^2 + b(n-1) + c] \) \\
    \( = (b+a) + (n-1) \cdot 2a \),
    
    Therefore, for \( n=1 \), \( a_1 = (b+a) + (1-1) \cdot 2a = b + a + c - S_1 \), \\
    and the general term is
    \[
    a_n = 
    \begin{cases}
        a + b + c & (n=1) \\
        (b+a) + (n-1) \cdot 2a & (n \geq 2)
    \end{cases}
    \]

    \item Since \( c = 0 \), \( a_n \) can be simplified to \( a_n = a + b \), which is constant and equals \( 2a \) when \( n \geq 2 \). This implies \( \{a_n\} \) is an arithmetic sequence with common difference \( 2a \), provided \( a, b \) are constants and \( a \neq 0 \).
    
    Note: From \( S_n \) we can deduce \( a_n = S_n - S_{n-1} \) when \( n \geq 2 \). Since \( a_1 = S_1 \), the sequence \( a_n = S_n - S_{n-1} \) (for \( n \geq 2 \)) and \( a_1 \) is the first term. The sequence \( \{a_n\} \) is an arithmetic sequence.
    
    Therefore, the general term \( a_n \) can be expressed as:
    \[
    a_n = 
    \begin{cases}
        S_1 & (n=1) \\
        S_n - S_{n-1} & (n \geq 2)
    \end{cases}
    \]
    
    Given the series \( \{a_n\} \) with \( S_n = an^2 + bm + c (a \neq 0) \), the sequence \( \{a_n\} \) is an arithmetic sequence with common difference \( 2a \) when \( c = 0 \).
\end{enumerate}

\begin{exercise}
    Given constants $a$, $b$, $c$, consider the sum $S_n = 1^2 + 2^2 + 3^2 + \ldots + n(n+1)^2 = \frac{n(n+1)}{12}(an^2 + bn + c)$, where $an^2 + bn + c \neq 0$.
\end{exercise}
\begin{proof}
    For $n=1$, we have $\frac{1}{6}(a+b+c)$, thus $a_1 = 4 = \frac{1}{6}(a+b+c)$.
For $n=2$, we have $\frac{22}{2} = 11 = \frac{1}{2}(4a+b+c)$, thus $a_2 = 22 = 9a + 3b + c$.
For $n=3$, $a_3 = 70 = 9a + 3b + c$.\\

From these equations, we find that:
\begin{align*}
    a + b + c &= 24 \\
    4a + b + c &= 44 \\
    9a + 3b + c &= 70
\end{align*}

Solving the system, we get $a=3$, $b=11$, $c=10$.
For \(n = 1, 2, 3\), the sum can be expressed as:
\[
1 \cdot 2^2 + 2 \cdot 3^2 + \ldots + n(n+1)^2 = \frac{n(n+1)}{12}(3n^2 + 11n + 10),
\]
thus, \(S_n = 1 \cdot 2^2 + 2 \cdot 3^2 + \ldots + n(n+1)^2\).

For a general term \(k\), \(S_k = \frac{k(k+1)}{12}(3k^2 + 11k + 10)\). Therefore,
\[
S_{k+1} = S_k + (k+1)(k+2)^2
\]
\[
= \frac{k(k+1)}{12}(3k^2 + 11k + 10) + (k+1)(k+2)^2
\]
\[
= \frac{k(k+1)}{12}((k+2)(3k+5) + (k+1)(k+2)^2)
\]
\[
= \frac{(k+1)(k+2)}{12}(3k^2 + 5k + 12k + 24)
\]
\[
= \frac{(k+1)(k+2)}{12}(3(k+1)^2 + 11(k+1) + 10).
\]

Hence, by induction, we can show that for \(n = k+1\) the sum is valid.

Finally, with \(a = 3\), \(b = 11\), \(c = 10\), we confirm that the given sequence is indeed a second-order arithmetic sequence.
\end{proof}

\begin{exercise}
    Evaluate:
    \begin{enumerate}
        \item $S = \sum_{1}^{n} \frac{N}{2^n}$
        \\
        \item $S = \sum_{1}^{n} \frac{3n-2}{5^{n-1}}$  
    \end{enumerate}
\end{exercise}

\textbf{Solution:}

(1) Given the series \( S_n = \frac{1}{2} + \frac{2}{4} + \frac{3}{8} + \cdots + \frac{n}{2^n} \), we can write:

\[
S_n - \frac{1}{2}S_n = \frac{1}{2} + \frac{1}{4} + \frac{1}{8} + \cdots + \frac{1}{2^n} - \frac{n}{2^{n+1}}
\]

This simplifies to:

\[
\frac{1}{2}S_n = \frac{1}{2} \left(1 - \left(\frac{1}{2}\right)^n\right) = \frac{1}{2} - \frac{1}{2^{n+1}} = \frac{1}{2} - \frac{n}{2^{n+1}} + \frac{n}{2^{n+1}}
\]

Hence, the series sum is:

\[
S_n = 2 - \frac{1}{2^{n-1}} - \frac{n}{2^n} = 2 - \frac{n+2}{2^n}
\]

(2) Considering the series \( S_n = 1 + \frac{4}{5} + \frac{7}{25} + \cdots + \frac{3n-2}{5^{n-1}} \), we proceed similarly:

\[
\left(1 - \frac{1}{5}\right)S_n = 1 + \frac{3}{5} + \frac{3}{25} + \cdots + \frac{3}{5^{n-1}} - \frac{3n-2}{5^n}
\]

The terms form a geometric series, so we get:

\[
S_n = 1 + \frac{3}{5} \left(1 + \frac{1}{5} + \frac{1}{25} + \cdots + \frac{1}{5^{n-2}}\right) - \frac{3n-2}{5^n}
\]

Applying the formula for the sum of a geometric series, we find:

\[
S_n = 1 + \frac{3}{5} \left(\frac{1 - \left(\frac{1}{5}\right)^{n-1}}{1 - \frac{1}{5}}\right) - \frac{3n-2}{5^n}
\]

Simplifying, we obtain:

\[
S_n = 1 + \frac{3}{5} \left(\frac{5^n - 5}{4 \cdot 5^{n-1}}\right) - \frac{3n-2}{5^n}
\]

Further simplification gives us:

\[
S_n = \frac{35}{16} - \frac{12n+7}{16 \cdot 5^{n-1}}
\]
%------------------------------------------------